\documentclass[11pt,a4paper,ngerman]{article}
\usepackage{babel}
\usepackage{fontenc}
\usepackage[left=3cm,right=2cm,top=2cm,bottom=3cm,includeheadfoot]{geometry}
\usepackage{xcolor}
\usepackage{float}
\usepackage{tikz}
\usepackage{minted}
\setminted{
	linenos,
	style=gruvbox-light,
	breaklines,
	frame=single,
	xleftmargin=0.1\textwidth,
	xrightmargin=0.1\textwidth,
	bgcolor=gray!10,
	fontsize=\footnotesize,
}
\renewcommand\listingscaption{Quellcode}
\title{Auftrag zu Mergesort}
\author{Olivier Warin}

\begin{document}
\maketitle
\tikzset{card/.style={rectangle,draw=black,inner xsep=0.5cm,inner ysep=0.5cm, thick,rounded corners,fill=gray!10,node distance=1.5cm}}
\newcommand\unsortedArray[1]{
	\begin{figure}[H]
		\centering
		\begin{tikzpicture}
			\node[card] (A)  {\(15\)};
			\node[card,right of=A] (B) {\(16\)};
			\node[card,right of=B] (C) {\(19\)};
			\node[card,right of=C] (D) {\(11\)};
			\node[card,right of=D] (E) {\(13\)};
			\node[card,right of=E] (F) {\(20\)};
			\node[card,right of=F] (G) {\(17\)};
			\node[card,right of=G] (H) {\(12\)};
		\end{tikzpicture}
		\caption{#1}
	\end{figure}
}
\def\numberOfCopies{4}
\begin{enumerate}
	\item  Sortieren Sie die folgende Liste sowohl mit Mergesort als auch mit Bubblesort.
		\unsortedArray{Zu sortierende Liste}
Zählen Sie bei beiden Verfahren, wie oft Sie zwei Zahlen miteinander vergleichen. \medskip

			\emph{Hinweis:} Schneiden Sie die Zahlen aus, um so einfacher Umordnen zu können. Sie finden daher hier noch \numberOfCopies~Kopien der Liste.
		\foreach \x in {1,...,\numberOfCopies} {\unsortedArray{Zu sortierende Liste (Kopie \x)}}
	\clearpage
	\item Implementieren Sie Mergesort mit Hilfe der Vorlage auf moodle.

		Vergleichen Sie die Laufzeiten von Mergesort und Bubblesort für zufällig generierte Arrays von den Längen 10, 1000 und 100000. Sie finden alle entsprechenden Funktionen in der Vorlage.
		\begin{listing}[H]
			\inputminted{python}{mergesort-vorlage.py}	
		\caption{Vorlage für Mergesort in Python, wie sie auf moodle zu finden ist.}
		\end{listing}
		
\end{enumerate}

% \begin{tikzpicture}
% 	\fill[black] (0,0) rectangle (16,9);
% 	\begin{scope}[shift={({(16-7*\distance)/2},6)}]
% 		\node[white,right] at (-1.7,1.5) {$\bullet$ Implementieren Sie Mergesort in Python.};
% 		\node[white,right] at (-1.7,0.8) {$\phantom{\bullet}$ Sie finden dazu eine Vorlage auf moodle.};
% 		\node[white,right] at (-1.7,0) {$\phantom{\bullet}$ Vergleichen Sie die Laufzeiten von};
% 		\node[white,right] at (-1.7,-0.7) {$\phantom{\bullet}$ Mergesort und Bubblesort.};
% 		\node[white,right] at (-1.7,-1.4) {$\phantom{\bullet}$ Sie finden die entsprechenden Funktionen };
% 		\node[white,right] at (-1.7,-2.1) {$\phantom{\bullet}$ in der Vorlage.};
% 	\end{scope}
% \end{tikzpicture}
\end{document}
